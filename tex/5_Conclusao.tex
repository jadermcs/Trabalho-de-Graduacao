Nesse trabalho, apresentei um método MtA baseado no \textit{framework} MetaStream. Melhorei o MetaStream pela adição de meta-atributos mais modernos e informativos e também pela inclusão do aprendizado incremental no nível MtA por meio do LightGBM como meta-classificador. Embora ambas as estratégias tenham performado de maneira similar, a estratégia incremental apresenta maior robustez devido seu menor tempo indutivo e uso de memória. Os resultados experimentais apresentaram que o meta-classificador pode consistentemente recomendar o melhor algorítimo para uma dada janela no fluxo de dados, conduzindo a um ganho acumulado de performance ao longo do tempo.


\section{Trabalhos Futuros}
\label{sec:future}
Como trabalhos futuros, gostaria de estender a classificação a mais de duas classes, isto é, uma classificação multi-classe, tornando possível selecionar entre múltiplos espaços de hipótese no nível base. Outra ideia que viável é usar de algoritmos incrementais para os meta-atributos, tornando possível diminuir o tempo de processamento, ou adicionar meta-atributos mais custosos porém mais informativos. Também, como proposto pelo artigo original \cite{rossi2014metastream}, atributos de séries temporais podem apresentar um grande poder discriminativo para esse tipo de problema. Uma abordagem recente é regredir modelos baseado em suas acurácias como forma de recomendação, o qual também a aplicável ao problema aqui proposto e permite estender a outros conjuntos de dados